\hypertarget{numeri-interi-modulo-e-segno}{%
\paragraph{Numeri interi (modulo e
segno)}\label{numeri-interi-modulo-e-segno}}

Abbiamo il numero \(125\).

Cambio di rappresentazione umano:

\begin{itemize}
\item
  da base 10 a base 2: \(125_{10}=1111101_{2}\).
\item
  da base 10 a base 8: \(125_{10}=175_{8}\).
\item
  da base 2 a base 16: \(1111101_{2}=7D_{16}\).
\end{itemize}

Codifica per l'utilizzo con un calcolatore a 16 bit (modulo e segno):

\begin{itemize}
\tightlist
\item
  da base 10 a base 2: \(+125 \rightarrow 0000000001111101_{2}\);
  \quad \(-125 \rightarrow 1000000001111101_{2}\).
\end{itemize}

Cambio di rappresentazione del codice di cui sopra:

\begin{itemize}
\item
  da base 2 a base 16: \(0000000001111101_{2}=007D_{16}\);
  \quad \(1000000001111101_{2}=807D_{16}\).
\item
  da base 2 a base 8: \(0000000001111101_{2}=000175_{8}\);
  \quad \(1000000001111101_{2}=100175_{8}\).
\end{itemize}

\emph{attenzione:} quando si cambia la rappresentazione del codice, non
bisogna mai dimenticare lo \emph{scopo} (rappresentare su 16 bit in
modulo e segno):

\begin{itemize}
\item
  lo riporto in binario (mantenendo 16 bit): ritorna esattamente lo
  stesso.
\item
  lo riporto in binario dall'ottale: ottengo 18 bit (6 cifre ottali),
  quindi devo scartare i 2 bit in più a sinistra e mantenere solo i 16
  bit meno significativi.
\end{itemize}

\begin{center}\rule{0.5\linewidth}{0.5pt}\end{center}

\hypertarget{numeri-interi-complemento-a-1}{%
\paragraph{Numeri interi (complemento a
1)}\label{numeri-interi-complemento-a-1}}

Abbiamo il numero \(125\).

Cambio di rappresentazione umano:

\begin{itemize}
\item
  da base 10 a base 2: \(125_{10}=1111101_{2}\).
\item
  da base 10 a base 8: \(125_{10}=175_{8}\).
\item
  da base 2 a base 16: \(1111101_{2}=7D_{16}\).
\end{itemize}

Codifica per l'utilizzo con un calcolatore a 16 bit (complemento a 1):

\begin{itemize}
\tightlist
\item
  da base 10 a base 2: \(+125 \rightarrow 0000000001111101_{2}\); \quad
  \(-125 \rightarrow \overline{0000000001111101}_{2}=1111111110000010_{2}\).
\end{itemize}

Cambio di rappresentazione del codice di cui sopra:

\begin{itemize}
\item
  da base 2 a base 16: \(0000000001111101_{2}=007D_{16}\);
  \quad \(1111111110000010_{2}=FF82_{16}\).
\item
  da base 2 a base 8: \(0000000001111101_{2}=000175_{8}\);
  \quad \(1111111110000010_{2}=177602_{8}\).
\end{itemize}

\emph{attenzione:} quando si cambia la rappresentazione del codice, non
bisogna mai dimenticare lo \emph{scopo} (rappresentare su 16 bit in
complemento a 1):

\begin{itemize}
\item
  lo riporto in binario (mantenendo 16 bit): ritorna esattamente lo
  stesso.
\item
  lo riporto in binario dall'ottale: ottengo 18 bit (6 cifre ottali),
  quindi devo scartare i 2 bit in più a sinistra e mantenere solo i 16
  bit meno significativi.
\end{itemize}

\begin{center}\rule{0.5\linewidth}{0.5pt}\end{center}

\hypertarget{numeri-interi-complemento-a-2}{%
\paragraph{Numeri interi (complemento a
2)}\label{numeri-interi-complemento-a-2}}

Abbiamo il numero \(125\).

Cambio di rappresentazione umano:

\begin{itemize}
\item
  da base 10 a base 2: \(125_{10}=1111101_{2}\).
\item
  da base 10 a base 8: \(125_{10}=175_{8}\).
\item
  da base 2 a base 16: \(1111101_{2}=7D_{16}\).
\end{itemize}

Codifica per l'utilizzo con un calcolatore a 16 bit (complemento a 2):

\begin{itemize}
\tightlist
\item
  da base 10 a base 2: \(+125 \rightarrow 0000000001111101_{2}\); \quad
  \(-125 \rightarrow \left(\overline{0000000001111101}_{2}+1\right)=1111111110000011_{2}\).
\end{itemize}

Cambio di rappresentazione del codice di cui sopra:

\begin{itemize}
\item
  da base 2 a base 16: \(0000000001111101_{2}=007D_{16}\);
  \quad \(1111111110000011_{2}=FF83_{16}\).
\item
  da base 2 a base 8: \(0000000001111101_{2}=000175_{8}\);
  \quad \(1111111110000011_{2}=177603_{8}\).
\end{itemize}

\emph{attenzione:} quando si cambia la rappresentazione del codice, non
bisogna mai dimenticare lo \emph{scopo} (rappresentare su 16 bit in
complemento a 2):

\begin{itemize}
\item
  lo riporto in binario (mantenendo 16 bit): ritorna esattamente lo
  stesso.
\item
  lo riporto in binario dall'ottale: ottengo 18 bit (6 cifre ottali),
  quindi devo scartare i 2 bit in più a sinistra e mantenere solo i 16
  bit meno significativi.
\end{itemize}
